\section{Literature Extending the Ben-Porath Framework}

\noindent In this sectio we summarize a variety of research that applies and extends the Ben-Porath model. The common feature of papers in this section is that they all address novel questions which have not been studied before and are frequently cited by other research. The purpose of this section is not to provide a comprehensive literature review. Instead, we aim to shed some light on how the Ben-Porath model can be employed and extended to study a wide range of research questions. Although the original Ben-Porath framework was developed to explain individuals' life cycle earnings profiles, future research adapts the framework to answer questions including life cycle investments in health, occupational choices, rising wage inequality, wage and promotion dynamics inside firms, child labor, childhood investments, wage growth of immigrants, gender gap in wage earnings, and sources of lifetime inequality. \\
\indent \citet{grossman1972concept} is the seminal theory paper studying people's life cycle investment decisions on health. Although health is considered an important dimension of human capital, \citet{grossman1972concept} argues that the existing human capital investment framework, including Ben-Porath, is not sufficient to capture people's demand for health. In Ben-Porath, an individual's incentive to invest in human capital is to increase her labor market productivity and thus her wage rate. However, the gain of having additional health is different. According to \citet{grossman1972concept}, individuals have two incentives to invest in health: (1) Health affects individuals' utilities directly; (2) Health determines the amount of time available for market and non-market activities in each period as well as the length of lifespan. Thus, \citet{grossman1972concept} adapts the Ben-Porath model to incorporate this feature of health investment.\\
\indent By using the adapted framework, \citet{grossman1972concept}  concludes that the demand for health capital and medical care are different. The model predicts that the demand for health capital would decline over the life cycle, whereas the demand for medical expenditure would rise with age. Both the demand for health and medical care are positively correlated with wage rate. The former is positively correlated with education, while the latter is negatively correlated with education.\\
\indent In addition to choosing schooling and on-the-job training as in the Ben-Porath model, \citet{keane1997career} also models individuals' choices on occupation. Unlike the Ben-Porath model, which focuses on a representative agent, individual's heterogeneity is explicitly modeled in \citet{keane1997career}. Individuals optimally choose general human capital investment (schooling) and occupation-specific human capital investment (occupation-specific work experience) based on their endowment heterogeneity, credit constraint conditions, and occupation-specific pricing of skills. The comparative advantage and sorting argument is applied in the selection process. \\
\indent To better fit the data on individual's schooling and occupational choices, \citet{keane1997career} extend the model even further. They include skill depreciation for non-work periods, which is also taken in account in Ben-Porath. They introduce market frictions, including job-finding costs and school reentry costs. Non-pecuniary components of occupational payoffs are proved to be empirically relevant.\\ %Don't even start a sentence with 'And'!! I tried to correct all of these but may have missed some.
\indent \citet{heckman1998explaining} develops a dynamic general equilibrium model with heterogeneous agents to explain the rising wage inequality in the U.S. The Ben-Porath model is extended in several ways. First, the original model does not distinguish human capital generated through schooling and human capital accumulated through on the job training. However, in \citet{heckman1998explaining}, schooling human capital is an input in the production of on the job training human capital. In particular, individuals with different levels of schooling could produce qualitatively different skills through on the job training. Second, skills corresponding to different levels of schooling have different prices. As in \citet{keane1997career}, this helps with introducing the story of comparative advantage and sorting. Moreover, \citet{heckman1998explaining} assumes that intial stocks of human capital are heterogeneous across individuals, and individuals have heterogeneous abilities to produce job-specific human capital, whereas the Ben-Porath model studies the behavior of a representative agent without introducing any individual heterogeneity. Finally, a general equilibrium model is constructed to capture the relationship between the capital market and the markets of different skill levels. \\
\indent Although many extensions are applied to the Ben-Porath model, one important feature is maintained, namely the skill prices and wage earnings are explicitly distinguished and allowed to move in different directions. With the general equilibrium framework, \citet{heckman1998explaining} is able to explain the main facts of U.S. wage inequality in the past 35 years.\\
\indent \citet{gibbons1999theory} aims at explaining a variety of empirical evidence on firms' wage and promotion dynamics. For example, the paper tends to justify the empirical finding that real wage decreases are not uncommon but demotions are very uncommon. Empirical findings, including that wage increases are serially correlated and promotions are associated with large wage increases, are also addressed. Moreover, the paper studies why workers receiving large wage increases early at a given job level are promoted faster.\\
\indent To explain these empirical findings systematically, the authors use the idea of the Ben-Porath model to incorporate individuals' on the job human capital acquisition behaviors. However, they also claim that using the Ben-Porath model alone is not sufficient. Two other principle elements are taken into account. In particular, they model how firms assign different jobs to different workers by using the idea of comparative advantage. They also include the component of learning, given that firms do not have perfect information on workers' productivity. By using the Ben-Porath model with two additional principal ingredients, the authors conclude that they are able to explain the main findings in the empirical literature on wage and promotion dynamics inside firms.\\
\indent \citet{baland2000child} provides a formal analysis on the issue of child labor. In their context, parents could decide how to allocate their child's time: either allowing her to study at school or requesting her to work as a child labor. Not surprisingly, the trade-off between child's working time and schooling time can be modeled by using a Ben-Porath type of framework.\\
\indent If the market were perfect, namely parents could borrow and lend as much as they want, leave debts to their child, and sign a formal contract with their child on the future returns of the human capital investments on their child, then parental decisions on child's education and labor force participation are proved efficient. However, if the market is imperfect, altruistic parents who face a binding budget constraint may choose to over spend their child's time on working instead of schooling. Then the inefficient amount of human capital investment in their child generates total social welfare loss.\\
\indent In developed countries where child labor barely exists, the opportunity cost of a child's time is irrelevant. Instead, the opportunity cost of the parent's time used in producing the child's human capital matters a lot. Accordingly, the Ben-Porath model can still be used to capture this trade-off. \citet{leibowitz1974home} is a pioneering study investigating the effect of family investments on child's future outcomes including ability, schooling and earnings. Following Ben-Porath, \citet{leibowitz1974home} applies a Cobb-Douglas human capital production technology, which assumes that the inputs, including the current level of human capital stock and investments, are complementary. This strong assumption is restrictive and has been relaxed in later research. \\
\indent In \citet{cunha2007technology} and \citet{cunha2010estimating}, empirical evidence is provided to show that the human capital formation process is governed by a multistage technology. Technologies change across different stages of the child development, and qualitatively different inputs can be used over time. The single stage Ben-Porath model oversimplifies the dynamics of the skill formation process.\\
\indent \citet{eckstein2004wage} studies the mechanism causing the differences in wage growth patterns between natives and immigrants in Israel from 1990-2000. Applying Ben-Porath's idea, the observed workers' wage earnings in the labor market are equal to their potential wage earnings minus the costs of investing in human capital acquisition. Assuming that the value of the imported skills brought by immigrants cannot be applied immediately after they arrive in the host country, then the price of the imported skills is  low at the beginning. This gives immigrants an incentive to invest more in local skills when they arrive, since the opportunity cost of investment is low. As the price of the imported skills rises over time since the arrival, the incentive to invest in local skills is reduced accordingly. Therefore, both the rising price of the imported skills and the investment incentives in local skills explain the steeper growth of the observed earnings profiles for the immigrants. The levels of the earnings for the immigrants also depend on the quality of the imported skills. With relatively low quality of the imported skills, as in the case discussed in \citet{eckstein2004wage}, the immigrants' earnings levels cannot catch up the natives' earnings levels.\\
\indent In \citet{manning2008gender}, the authors tend to solve the puzzle that whereas the average earnings of the males and females are similar when they enter the labor market, the growth of the earnings for males is much faster than for females in the first ten years after labor market entry. The Ben-Porath model provides an answer to justify the puzzle. In particular, if women expect to work less or even leave the labor market in order to specialize more in the household production after marriage, they have less incentive to take on the job training and thus their earnings profile grows less. However, by using a U.K. data-set, the authors find that the human capital investment explanation can only capture 50\% of the gender wage gap.\\
\indent Moreover, the authors test two other explanations. The first one assumes that women are less concerned with money but more concerned with other non-pecuniary issues when they seek to change jobs, which leads to lower returns for job changes for women. The second explanation uses the psychological differences between men and women to justify the differences in their wage growth. It has been well documented that men and women are very different in terms of attitudes to risk-taking, competition, self-esteem, and selfishness. Nevertheless, it is shown that empirically, both the job shopping and psychological explanations can explain only a small percentage of the gender wage gap. Thus a sizable proportion of the gender wage difference remains unexplained.\\
\indent \citet{Huggett2011sources} documents that the mean of individuals' earnings is hump-shaped over the life cycle and the dispersion of individuals' earnings is increasing with age. The authors justify the hump-shaped mean earnings profile simply by using the Ben-Porath model. To justify the increasing pattern of the dispersion, the authors add features to the Ben-Porath framework, including idiosyncratic human capital production shocks, different amounts of human capital stocks at age 23, which is the starting age of the life cycle earnings profiles studied in this paper, and differences in individuals' learning ability and wealth. The show that differences in individuals' learning ability explain a large part of the rising dispersion.\\
\indent The estimated Ben-Porath framework with a risky human capital production technology shows that the differences in individuals' lifetime earnings are mainly due to the differences in individuals' initial conditions at age 23, rather than the idiosyncratic shocks experienced over the rest of their lives. And among the initial conditions, the variations in the initial human capital stocks are shown to be much more relevant than the variations in individuals' learning abilities and wealth for explaining the dispersion in individuals' lifetime earnings. \\
\indent In this section, a variety of research which addresses different research questions by using the Ben-Porath framework is summarized. Although it is natural to extend the basic framework to incorporate additional features when tackling different questions, the fundamental idea behind the original framework remains in all these studies. In particular, the essential trade-off faced by decision makers when choosing the optimal amount of human capital investment is to compare the future gains from having an additional stock of human capital and the opportunity costs of making such an investment. Although the form and dynamic of the relevant benefits and costs vary between different contexts, leading to different conclusions on the trajectory of human capital investment, we can see that since the same basic trade-off underlies all the research topics summarized above, the Ben-Porath framework has a rather wide usage and still has potential to be further extended to solve new research questions in the future. \\