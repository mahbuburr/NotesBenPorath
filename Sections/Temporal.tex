\section{The Rate of Return of Post-school Investment}
We analyze the rate of return of post-school investment based on the model in Section \ref{section:baseline}. Recall that this model considers no depreciation, $\sigma = 0$, and $T \rightarrow \infty$.\footnote{Recall that this implies that $g(t) = \frac{R}{r}$}. Now, let $E(\tau)^{NPS}$ and $E(\tau)^{PS}$ denote earnings without and with post-schooling investment, respectively. By \eqref{eq:earnsall} we can write
\begin{eqnarray}
E(\tau)^{NPS} &=& R H(t*) \\ \nonumber
&=& R \left( \frac{\alpha A}{r} \right)^{\frac{1}{1 - \alpha}} \\ \nonumber
E(\tau)^{PS} &=& \left[ \frac{\alpha A}{r} \right]^{\frac{\alpha}{1-\alpha}} \tau
\end{eqnarray}

\noindent so that the increment in earnings due to post-schooling at $\tau$ is

\begin{eqnarray}
\Delta^{E(\tau)} \equiv E(\tau)^{NPS} - E(\tau)^{PS}.
\end{eqnarray}

\indent Actually, is we note that $E(\tau)^{PS} = IH(\tau)$ (see \eqref{eq:itcobb}) then we can interpret $\Delta^{E(\tau)}$ as ``returns less costs'' from post-schooling and let $\phi$ be the rate of return of post-schooling be defined by